\documentclass[../main.tex]{subfiles}
\begin{document}

\section{Summary}

This thesis explores the capability of deep learning models in predicting \gls{AKI} in patients with \gls{DKA}.
DKA is a serious complication of diabetes that can lead to AKI, a sudden decline in kidney function.
AKI, a severe complication that affects about 40\% of \gls{DKA} patients, can significantly increase morbidity and mortality.
Early detection of AKI is essential for improving patient outcomes and reducing mortality rates.
However, current prediction models utilizing static data from the MIMIC-IV database have limitations.

This thesis addresses these limitations by proposing a novel approach to predict AKI in DKA-ICU patients.
The primary goal is to develop a model that leverages data from the MIMIC-IV database to achieve more accurate predictions.
MIMIC-IV is a rich repository of de-identified health records, making it a valuable resource for studying AKI and other critical conditions.
However, existing studies only focus on machine learning models and static data, which may not capture the dynamic nature of patient health metrics.
Additionally, current models may include features that cause data leakage, essentially giving the model information about the outcome it's trying to predict.
This thesis proposes removing such features to ensure the model's predictions are based solely on relevant clinical and biological data.

Additionally, the thesis explores the impact of using different methods for calculating various patient indicators.
Some indicators may have multiple measurements recorded over time.
Analyzing how these calculations are done and their influence on the model's performance is crucial for optimizing its effectiveness.

A significant contribution of this thesis lies in the development and evaluation of a deep learning model that utilizes time-series data.
Deep learning algorithms are particularly adept at identifying complex patterns in data, making them well-suited for analyzing the dynamic nature of patient health metrics.
By incorporating time-series data and leveraging the power of deep learning, this model has the potential to outperform existing prediction models in terms of accuracy and reliability.

The study focuses on developing and comparing multiple deep learning models, including tabular-based and time series-based approaches, to predict the risk of \gls{AKI} within the first 7 days of \gls{ICU} admission for \gls{DKA} patients.
The models were trained using various features collected 24 hours prior to \gls{ICU} admission, such as demographics, vital signs, comorbidities, testing results, interventions, prognosis, and scoring systems.
Patient's data were extracted from the \gls{MIMIC-IV} database, a publicly available dataset of critical care patients via SQL queries.
They were then preprocessed and transformed into a more robust format where each patient's data was represented as an object.

The Patient object contains collected features, which are either static data which do not change over the course of ICU stay or time series data stored as a map of charted time and the corresponding value.
The object has multiple methods supporting the analysis of the patient's data, such as counting amount of existing features, extract features during a time window.
These information were then used to train the deep learning models, including \gls{LSTM}, TabPFN, and GRANDE.
Finally, the models were evaluated based on their performance metrics, including accuracy, sensitivity, specificity, and \gls{AUC}.


\section{Remarkable contributions}

The study has several notable contributions to the field of AKI prediction in DKA patients. 
Firstly, the study proposes a novel approach to predict AKI in DKA patients using deep learning models.
The models leverage time-series data to capture the dynamic nature of patient health metrics, shredding light on the potential of deep learning in improving AKI prediction accuracy.

Secondly, the study introduces a new method for calculating patient indicators, which may have multiple measurements recorded over time.
By analyzing the impact of different calculation methods on the model's performance, the study provides valuable insights into optimizing the model's effectiveness.
It is found that using the latest value of an indicator at the time of prediction yields the best results, highlighting the importance of selecting the appropriate calculation method.

Thirdly, the study removes features that may cause data leakage while adding new features supporting the prediction of AKI in DKA patients. 
Furthermore, the study also identified and deleted the polluted data measured after the development of AKI, which may lead to data leakage.
Overall models' AUC scores improved for about 2\%.

Lastly, the study develops and evaluates multiple deep learning models, including tabular-based and time series-based approaches, to predict the risk of AKI in DKA patients.
The models are trained using various features collected 24 hours prior to ICU admission, providing a comprehensive analysis of the patient's condition and risk factors.
Final result shows that the TabPFN model outperforms the other models in terms of accuracy, sensitivity, specificity, and AUC, demonstrating its potential for improving AKI prediction in DKA patients.


\section{Suggestion for Future Works }

Although having achieved promising results, this study has several limitations that could be addressed in future research.
Firstly, the dataset used in this study was collected from a single source, which may not be representative of the general population.
Future studies should consider using a more diverse dataset to improve the generalizability of the models.

Secondly, the study focused on predicting \gls{AKI} within the first 7 days of \gls{ICU} admission.
This means \gls{AKI} high risk patients who develop the condition in normal hospital wards are not considered.
Future research could explore the prediction of \gls{AKI} in a broader context, including patients outside the ICU.

Lastly, the study utilized a limited set of features collected 24 hours prior to \gls{ICU} admission.
Which limits the model's ability to capture the full development of the patient's condition.
Future research could aim to promptly predict \gls{AKI} with real-time data, providing more accurate and timely predictions.

\end{document}