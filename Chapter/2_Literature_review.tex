\documentclass[../main.tex]{subfiles}
\begin{document}

\par In this chapter, I will conduct a literature review on the topic of predicting Acute Kidney Injury, especially in patients with Diabetic Ketoacidosis admitted to ICU. 
The purpose of this chapter is to provide an overview of the current state of research on the topic, as well as to identify the gaps in the existing literature that this thesis aims to address.


\section{Scope of Research}
The scope of this chapter is centered on predicting AKI in patients with Diabetic Ketoacidosis  during their ICU admission, utilizing data from the MIMIC-IV database. 
The emphasis lies on the development of novel predictive models and data engineering strategies tailored specifically for this population.
Thus, the focus will be on introducing innovative methodologies rather than extensively examining the relationships between AKI indicators or expanding the dataset.
In addition, the scope is limited current AKI definition by KDIGO guidelines, which is based on the changes in serum creatinine and urine output.


\section{Related Works}
% Trong phần này sinh viên trình bày các nghiên cứu liên quan (related work), chú ý phân tích rõ những ưu nhược điểm của chúng. Từ đó, nêu bật lên động lực để thực hiện nghiên cứu của đồ án này.

Several studies have explored the prediction of Acute Kidney Injury in critical care settings, providing valuable insights into the development of predictive models and methodologies. 

Firstly, the KDIGO guidelines offer standardized criteria for detecting AKI, serving as a foundational framework for AKI prediction studies.
The definition of AKI provided by KDIGO is based on the changes in serum creatinine and urine output, which are described as patients who satisfy at least one of the following criteria:

\begin{equation}
    \text{ Increase in SCr by } \geq 0.3 \text{ mg/dL } \text{ within 48 hours}
\end{equation}

\begin{equation}
    \text{ Increase in SCr to } \geq 1.5 \text{ times baseline }
\end{equation}
The times baseline is known or presumed to have occurred within the prior 7 days.

\begin{equation}
    \text{ Urine volume } \le 0.5 \text{ ml/kg/h for 6 hours}
\end{equation}

Although the KDIGO guidelines provide a standardized definition of AKI, upon detecting AKI, the damage has already occurred due to either high levels of creatinine or accumulation of waste products in the blood.
And the need for early prediction of AKI is demanding to prevent further complications and improve patient outcomes.

While other studies struggled with the subtle nature of AKI, a study by \cite{chen2021nomogram} has developed a nomogram to predict the risk of Acute Kidney Injury for KDA-ICU patients.
This has shed light for the following researches as ICU patients got monitored more frequently than general patients, which could provide more accurate and abundant data for the prediction model.
On the flip side, the study used a deprecated definition of AKI and an older MIMIC version, leading falsely labeled AKI patients and small dataset.
Their best model, achieved an AUC of 0.747, which is a promising result but still has room for improvement.

Another study by \cite{goceri2021predicting} has proposed using predictive models based on a machine learning approach to identify patients with DKA at increased risk of AKI within 1 week of hospitalization in the intensive care unit.
With state-of-the-art machine learning algorithms, the study achieved an AUC of 0.800, which is a significant improvement compared to the previous study.
However, the study did not take into account the changes in patients' indicators over time, which could be crucial for predicting Acute Kidney Injury as the disease progresses.


\section{Tên của kiến thức nền tảng số 1}
Tiêu đề và nội dung của chương này sẽ thay đổi tuỳ thuộc vào từng đồ án. Chú ý trình bày những kiến thức có liên quan mật thiết nhất đối với đồ án của mình, Tránh trình bày lan man những kiến thức phổ thông không cần thiết. 

\section{Tên của kiến thức nền tảng số 2}
Tiêu đề và nội dung của chương này sẽ thay đổi tuỳ thuộc vào từng đồ án. Chú ý trình bày những kiến thức có liên quan mật thiết nhất đối với đồ án của mình, Tránh trình bày lan man những kiến thức phổ thông không cần thiết. 

\end{document}