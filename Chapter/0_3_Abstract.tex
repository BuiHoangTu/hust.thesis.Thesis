\documentclass[../main.tex]{subfiles}
\begin{document}

\begin{center}
    \Large{\textbf{ABSTRACT}}\\
\end{center}
\vspace{1cm}
% Sinh viên viết tóm tắt ĐATN của mình trong mục này, với 200 đến 350 từ. Theo trình tự, các nội dung tóm tắt cần có: (i) Giới thiệu vấn đề (tại sao có vấn đề đó, hiện tại được giải quyết chưa, có những hướng tiếp cận nào, các hướng này giải quyết như thế nào, hạn chế là gì), (ii) Hướng tiếp cận sinh viên lựa chọn là gì, vì sao chọn hướng đó, (iii) Tổng quan giải pháp của sinh viên theo hướng tiếp cận đã chọn, và (iv) Đóng góp chính của ĐATN là gì, kết quả đạt được sau cùng là gì. Sinh viên cần viết thành đoạn văn, không được viết ý hoặc gạch đầu dòng.

\par Diabetic Ketoacidosis (DKA) is a serious acute complications of Diabetes Mellitus that can be life-threatening if not treated promptly.
DKA can lead to other complications such as Cerebral Edema, Acute Kidney Injury (AKI), or in severe cases, kidney failure.
Among them, AKI is a common complication, affecting about 40\% of DKA patients.
Furthermore, AKI increases mortality and morbidity from other diseases, leading to prolonged treatment time in the intensive care unit (ICU).
This complication has a high recurrence rate during ICU treatment and can result in Chronic Kidney Disease (CKD).
Usually, upon detected, Acute Kidney Injury has already make a significant impact on the patient's health.

\par Predicting the risk of AKI in DKA patients is essential for early intervention and treatment, helping minimize negative effects.
Several risk factors of developing AKI for DKA patients have been investigated over the past decade, including older age, increased glucose, serum uric acid, white blood, \dots
In previous studies, models have been applied to predict the risk of AKI in DKA patients using logistic regression or machine learning algorithms.
However, according to my knowledge, there is currently no models taking the changes in patients' indicators overtime into account and the power of deep learning is still underexplored.

\par Therefore, in this thesis, I propose a deep learning model to predict the risk of AKI in DKA patients during their ICU stay.
Multiple tabular-based deep learning models and time series-based deep learning models are implemented and compared with traditional machine learning models.
The models are trained to predict if a DKA patient will develop AKI within the next 7 days after ICU admission.

The following features are collected 24 hours prior to the admission time: demographics, vital signs, comorbidities, testing, interventions, prognosis, and scoring systems.

\begin{flushright}
\begin{tabular}{@{}c@{}}
Student\\
\textit{(Signature and full name)}
\end{tabular}
\end{flushright}

\end{document}