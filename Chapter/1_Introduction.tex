\documentclass[../main.tex]{subfiles}
\begin{document}


\section{Problem Statement}
\label{sec:dvd}

\gls{DKA} is a metabolic complication of \gls{DM}, caused by insulin deficiency and increased counterregulatory hormones.
This results in the body not converting enough sugar into energy and requiring to metabolize triglycerides.
Overtime, this process produces free fatty acids which are converted into ketone bodies in the liver.
The evaluation of FFAs exacerbates insulin resistance and hyperglycemia, creating a malignant cycle \cite{dka-info}.
In Vietnam, the number of diabetes ketoacidosis patients were reported at 3.53 million in 2017 \cite{vn-dka-3.5-2019} and were predicted to rise to 6.3 million in 2045 \cite{vn-dka-predict-6.3-2045}.
However, the actual number was doubled in the past 6 years to 7 million people in 2023 \cite{vn-dka-7-2023}.
Vietnamese Ministry of Health acknowledge that \gls{DKA} is in top 3 of the most non-communicable diseases \cite{vn-dka-predict-6.3-2045}.
It is worth noting that in 7 million patients, 55\% developed complications 34\% of which is cardiovascular diseases; 39.5\% had eyes complications; 24\% had kidney complications \cite{vn-dka-7-2023}.
A study conducted in Australia and New Zealand of 8533 \gls{DKA} patients illustrated that the rate of \gls{DKA} admission to intensive care unit has increased fivefold over the past decade \cite{aus+nz-dka-icu-increase}.
This makes complications of Diabetic Ketosis more common, especially Acute kidney injury.

\gls{AKI} is a common complication of hospitalized patients, with a high mortality rate.
It occurs in approximately 30-50\% of \gls{ICU} patients.
AKI is characterized as a sudden worsen of the kidney's blood filtering function, usually caused by a buildup of waste in the blood that prevents the kidneys from balancing the body's electrolytes.
This complication can develop in just a few hours or a few days.
In most cases, it occurs within 48 hours but can last up to 7 days.
AKI can cause serious complications such as chronic kidney disease or damage to other organs in the body such as the brain, heart and lungs.
Often when acute kidney injury is detected, it has left severe sequelae due to waste accumulation in the blood affecting organs in the body, increasing mortality, treatment time and cost \cite{kdigo-aki-guideline}.
In addition, due to damage to the kidneys caused by AKI, patients face a greater susceptibility of disease recurrence, especially in patients with underlying diseases that put pressure on the kidneys such as \gls{DKA} patients.

Therefore, predicting \gls{AKI} in advance is an urgent issue in improving patient care, allowing doctors to intervene promptly to stop disease progression.



\section{Background and Problems of Research} 
\label{sec:giaiphap}

%%% AKI explain %%%
\gls{AKI} is a sudden episode of kidney failure or damage that happens within a few hours or days. 
AKI leads to a buildup of waste products in the blood, making it difficult for the kidneys to maintain the right balance of fluids in the body. 
This condition can also affect other organs, including the brain, heart, and lungs. Due to its critical nature, \gls{AKI} requires quick detection and constant management to prevent long-term consequences and improve patient outcomes.

AKI can result from various causes, typically categorized into three groups: pre-renal, intrinsic renal, and post-renal. 
Pre-renal causes include conditions that reduce blood flow to the kidneys, such as severe dehydration, blood loss, or low blood pressure. 
Heart failure or liver cirrhosis can also lead to pre-renal AKI.
Intrinsic renal causes are conditions that directly damage the kidneys, such as glomerulonephritis, acute tubular necrosis, or interstitial nephritis, often due to nephrotoxic drugs, infections, or autoimmune diseases. % kidney toxic drugs, infections, or autoimmune diseases.
Post-renal causes involve obstructions that block urine flow out of the kidneys, such as kidney stones, tumors, or an enlarged prostate.
 
Early detection of \gls{AKI} is crucial for timely intervention and management, potentially reversing the condition before it causes irreversible kidney damage. 
The early signs of \gls{AKI} can be subtle and non-specific, making awareness and prompt diagnosis essential. 
Clinical evaluation and history are vital, especially for patients with risk factors such as major surgery, critical illness, or existing kidney disease. 
Symptoms like reduced urine output, swelling in legs, ankles, or around the eyes, fatigue, confusion, nausea, or chest pain should be closely monitored.

Laboratory tests and biomarkers play a significant role in early detection. 
Serum creatinine is a key indicator, with even a small rise suggesting kidney dysfunction. 
Elevated Blood Urea Nitrogen (BUN) levels can also indicate kidney problems, while monitoring urine output can reveal early signs of \gls{AKI} through oliguria. 

In conclusion, \gls{AKI} is a serious medical condition with significant morbidity and mortality. 
Early detection through clinical vigilance, the use of biomarkers, and imaging studies is essential for prompt management and improving patient outcomes. 
Preventative strategies and careful monitoring of at-risk individuals can significantly reduce the incidence and impact of AKI. 
Advances in biomarkers and diagnostic techniques hold promise for even earlier detection and intervention, ultimately improving the prognosis for patients with AKI.


%%% \gls{DKA} explain %%%
\gls{DKA} is a serious and potentially life-threatening complication of \gls{DM}, characterized by hyperglycemia, ketosis, and metabolic acidosis. 
DKA commonly occurs in individuals with Type 1 diabetes but can also present in those with Type 2 diabetes, particularly under conditions of severe stress, infection, or inadequate insulin therapy. 

Diabetic Ketoacidosis arises from a severe deficiency of insulin, leading to hyperglycemia and the accumulation of ketone bodies due to increased fatty acid oxidation. 
The primary precipitating factors include infection, missed insulin doses, and new-onset diabetes. 
Without sufficient insulin, glucose cannot enter cells for energy, prompting the body to break down fats as an alternative energy source. 
This process produces ketones, which are acidic by-products that accumulate in the blood, leading to metabolic acidosis. 
The hallmark signs of \gls{DKA} include high blood glucose levels (hyperglycemia), the presence of ketones in the urine and blood (ketonemia), and acidosis.

It can significantly contribute to the onset and progression of \gls{AKI} through several mechanisms such as Dehydration and Volume Depletion, Electrolyte Imbalances, Acidosis, Inflammatory Response and Oxidative Stress. 
Severe dehydration due to osmotic diuresis is a primary factor in \gls{DKA}. 
The hyperglycemic state causes an increase in osmotic pressure, leading to significant fluid loss through the kidneys. 
This profound volume depletion reduces renal perfusion, potentially precipitating pre-renal AKI. 
If dehydration is not promptly corrected, it can progress to intrinsic renal damage.
In addition, \gls{DKA} is associated with significant electrolyte disturbances, including hyperkalemia and hyponatremia. 
The shifting of potassium from the intracellular to the extracellular space, followed by renal losses, can lead to significant potassium depletion. 
Electrolyte imbalances can exacerbate renal dysfunction, further contributing to AKI
The metabolic acidosis seen in \gls{DKA} can also have direct toxic effects on the kidneys. 
Acid-base disturbances alter renal blood flow and glomerular filtration rate, leading to further kidney injury. 
Acidosis also affects the tubular cells, making them more susceptible to damage and necrosis.
Furthermore, Diabetic Mellitus triggers a systemic inflammatory response and increases oxidative stress. 
The release of pro-inflammatory cytokines and reactive oxygen species can lead to endothelial dysfunction and tubular injury, thereby promoting AKI.

In short, \gls{DKA} is a serious complication of diabetes that can significantly impact renal function and lead to Acute Kidney Injury. 
The pathophysiological mechanisms linking \gls{DKA} and \gls{AKI} include dehydration and volume depletion, electrolyte imbalances, metabolic acidosis, and systemic inflammation. 
Early recognition and prompt management of \gls{DKA} are vital to prevent the development of \gls{AKI} and other complications.

Correcting hyperglycemia, dehydration, and electrolyte imbalances have gigantic effect on preventing the onset of AKI. 
Once the risk was detected, the patient can be closely monitored for signs of \gls{AKI} and treated promptly with Fluid Resuscitation, Insulin Therapy, Electrolyte Management. 
Fluid replacement help to restore intravascular volume and renal perfusion, while insulin therapy corrects hyperglycemia and suppresses ketogenesis.
Regular monitoring and correction of electrolyte imbalances are essential to relieve the pressure on the kidneys and prevent further renal damage.


That is why, several studies have been carried out to predict the risk of \gls{AKI} in \gls{DKA} patients aiming to improve patient outcomes and reduce the burden of \gls{AKI} on healthcare systems.
Many have made significant advances in predicting \gls{AKI} in \gls{DKA-ICU} patients using mainly tabular-based machine learning models.
These studies shed light on a new direction for predicting the risk of developing Acute Kidney Injury for \gls{DKA-ICU} patients alone rather than for all ICU-admitted patients.
In more recent study, the researchers proposed clinical and biological features, including biological indicators such as creatinine, urea, glucose, etc. and clinical indicators such as blood pressure, heart rate, etc., to build an \gls{AKI} prediction model using machine learning algorithms showed that the model's ability to predict \gls{AKI} was more optimal than previous models with AUCs of 80\% and 74.7\% respectively.

Although they have shown success in predicting patients with Acute Kidney Injury, these models still have some limitations and shortcomings.
In both models, the data used to train the model is only static data, does not account for indicators changes over time.
This cripples the model's performance in the patients whose indicators change drastically after the initial observation of which the data is used to train the model.

In addition, the model uses some unreasonable indicators which were not available before the development of \gls{AKI} including mortality rate, interventions specific to Acute Kidney Injury.
These features cause data leakage with the model, leading to inaccurate assessments with part of the data.

Furthermore, the dataset only includes data from the \gls{MIMIC-IV} database, so the patient set is small and not representative of all \gls{DKA-ICU} patients.

Finally, the method for assessing \gls{AKI} in patients is not yet complete and updated regularly, so there exist patients falsely labeled in previous studies.

Overall, although there have been noticeable advances in predicting \gls{AKI} in \gls{DKA-ICU} patients, there are still many limitations that need to be overcome so that the model become stable, reliable, and suitable to more general usecase.



\section{Research Objectives and Conceptual Framework}

The main goals of this thesis is to find out the most suitable model to predict \gls{AKI} in \gls{DKA-ICU} patients based on clinical and biological data from \gls{MIMIC-IV} database.

To begin, the first objective needs to address the static features used  in current models which require me to delve into the \gls{MIMIC-IV} database to gain a comprehensive understanding of its data structure including the structure of each table, the relationship between tables and how \gls{MIMIC-IV} store data.
In this section, it is important to remove indicators that may cause data leakage and add indicators commonly observed in \gls{DKA-ICU} patients which was ignored in previous studies.

The second goal focuses on designing and implementing a data structures that has ability to accommodate patient metrics' changes over time.
To accomplish this, I choose to save the indicators in a JSON-like structure so a data preprocessing pipeline was built to convert data from tabular form to JSON-like format.

Thirdly, I aim to replicate the \gls{AKI} tabular-based prediction model in \gls{DKA-ICU} patients that had been performed in previous studies to ensure that the models in the old study and the new models are evaluated on the same data set.

Finally, I evaluate and compare the predictive ability of different models in predicting Acute Kidney Injury in \gls{DKA-ICU} patients.
The models compared include XGBoost (Best Model in Previous Studies), new deep learning models with tabular data, and deep learning models with time series data.

\section{Contributions}

This thesis' contributions can be summarize into 4 main points.

Firstly, it proposes the removal of certain data-leaking indices and the addition of commonly measured metrics in \gls{DKA-ICU} patients to predict AKI. 
By identifying and eliminating indices that contribute to data leakage, the accuracy and reliability of the predictions are enhanced. 
Additionally, the inclusion of commonly measured indices expect to boost models' performance.

Secondly, this thesis updates the method for identifying \gls{AKI} patients. This improvement involves employing latest algorithms utilized to indicate Acute Kidney Injury, ensuring that the identification process is more accurate.
By enhancing the identification method, the project aims to improve early detection and intervention for patients at risk of AKI, ultimately leading to better outcomes.

Thirdly, it evaluates the predictive capability of tabular-based models with different indicators calculation methods. 
This involves a thorough analysis of various methods for calculating indicators if several ones were measured and their impact on the performance of predictive models. 
By comparing these methods, the project identifies the most effective approaches for accurately predicting AKI, providing valuable insights for future research and clinical practice.

Finally, the main contribution of the thesis lies in developing and evaluating a deep learning model with time-series data.
This innovative approach leverages the power of deep learning to analyze patterns and trends in time-series data, offering a sophisticated tool for predicting AKI.

\section{Organization of Thesis}
The rest of the thesis report is organized into five main chapters.

Chapter 2 provides a comprehensive review of existing literature pertaining to the prediction of Acute Kidney Injury,  which consist of its pathophysiology, risk factors, and clinical outcomes.
By delving into past research, I will explore the various predictive models, risk factors, and early biomarkers identified over the years.
This examination will shed light on how early prediction of \gls{AKI} has evolved, highlighting key advancements and methodologies that have contributed to improved predictive accuracy.
Understanding these foundational studies will provide valuable insights into the development of current prediction strategies and help identify potential areas for further research and innovation in forecasting AKI.
In addition, this chapter will also review the literature background of models and techniques used in predicting \gls{AKI} in \gls{DKA-ICU} patients, providing a basis for the subsequent chapters.

In Chapter 3, I illustrate the process of extracting and preparing data from the \gls{MIMIC-IV} database for predicting Acute Kidney Injury in Diabetic Ketoacidosis patients in the ICU.
This includes the inspection the Medical Information Mart for Intensive Care IV (MIMIC-IV) database to establish an method to extract the desired data for predicting AKI. 
MIMIC-IV, with its rich repository of de-identified health records, offers a valuable resource for studying Acute Kidney Injury and other critical conditions. 
I will detail the process of navigating the database structure, identifying relevant data fields, and querying the database to retrieve specific datasets. 
Additionally, I will outline the steps involved in building a robust data pipeline to ensure efficient storage, retrieval, and management of the extracted data. 
This pipeline will incorporate best practices for data cleaning, transformation, and integration, facilitating seamless downstream analysis and research. 
This gives a clear framework for leveraging \gls{MIMIC-IV} data in \gls{AKI} prediction studies will be established, paving the way for more effective and data-driven medical research.
By the end of this chapter, I will provide the approach to each models in order to gain the best performance.

Finally, in Chapter 5, I will compare the results of the built models and evaluate their predictive ability. 
These models will be assessed using various model evaluation metrics, such as the \gls{AUC}, precision, accuracy and recall. 
By benchmarking these metrics, I will compare the performance of our models against those from previous studies. 
This comparative analysis will highlight the strengths and weaknesses of each model, providing insights into their effectiveness in predicting \gls{AKI} in \gls{DKA} patients in the ICU. 
Additionally, I will discuss any observed improvements or shortcomings, offering explanations for the differences in performance. 
This chapter aims to provide a comprehensive evaluation, ensuring that the most accurate and reliable model is identified for practical implementation in clinical settings.


\section{Conclusion}
In this chapter, I have presented an overview of the project including the problem, goals and solution orientation along with the project's contributions and the project's layout.
In the next chapters, I will go deeper into each part of the project to solve the problems mentioned in this chapter.


\end{document}