\documentclass[../main.tex]{subfiles}
\begin{document}


\begin{table}
\centering
\caption{Features of patients}
\begin{tabular}{|l|p{0.6\textwidth}|}
    \hline
    \textbf{Category} & \textbf{Variable} \\
    \hline

    Demographic & 
    Age , gender , ethnicity, height, weight \\
    \hline

    Vital signs & 
    HR, RR, SBP, DBP \\
    \hline

    Characteristics of diabetes &
    DM type, microangiopathy, macroangiopathy \\
    \hline
    
    Comorbidities &
    history of AMI, history of ACI, CHF, liver disease, preexisting-CKD, malignant cancer, hypertension, chronic pulmonary disease \\
    \hline

    Laboratory test &
    WBC, lymphocyte, Hb, PLT, PO2, PCO2, PH, AG, bicarbonate, BUN, calcium, Scr, BG, phosphate, albumin, eGFR, HbA1C, CRP, urine ketone, \textbf{hematocrit, mch, mchc,mcv, rbc, rdw} \\
    \hline

    Scoring systems &
    GCS, OASIS, SOFA, SAPSII, \textbf{chloride, sodium, potassium} \\
    \hline

    Interventions &
    MV, use of NaHCO3 \\
    \hline

    Prognosis &
    PreIcuLos \\
    \hline

\end{tabular}

\end{table}

This table shows the features of patients in the dataset. 
The bolded variables are the new features that we have added to the dataset.
Microangiopathy means patients with diabetic nephropathy, diabetic retinopathy, or diabetic peripheral neuropathy. Macroangiopathy means patients with coronary heart disease, cerebral atherosclerosis, or peripheral atherosclerosis.


\end{document}