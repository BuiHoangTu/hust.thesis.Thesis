\documentclass[../main.tex]{subfiles}
\begin{document}

\par In this chapter, I will conduct a literature review on the topic of predicting Acute Kidney Injury, especially in patients with Diabetic Ketoacidosis admitted to ICU. 
The purpose of this chapter is to provide an overview of the current state of research on the topic, as well as to identify the gaps in the existing literature that this thesis aims to address.


\section{Scope of Research}
The scope of this chapter is centered on predicting AKI in patients with Diabetic Ketoacidosis  during their ICU admission, utilizing data from the MIMIC-IV database. 
The emphasis lies on the development of novel predictive models and data engineering strategies tailored specifically for this population.
Thus, the focus will be on introducing innovative methodologies rather than extensively examining the relationships between AKI indicators or expanding the dataset.
In addition, the scope is limited current AKI definition by KDIGO guidelines, which is based on the changes in serum creatinine and urine output.


\section{Related Works}

Several studies have explored the prediction of Acute Kidney Injury in critical care settings, providing valuable insights into the development of predictive models and methodologies. 

Firstly, the KDIGO guidelines offer standardized criteria for detecting AKI, serving as a foundational framework for AKI prediction studies.
The definition of AKI provided by KDIGO is based on the changes in serum creatinine and urine output, which are described as patients who satisfy at least one of the following criteria:

\begin{equation}
    \text{ Increase in SCr by } \geq 0.3 \text{ mg/dL } \text{ within 48 hours}
\end{equation}

\begin{equation}
    \text{ Increase in SCr to } \geq 1.5 \times \text{ baseline }
\end{equation}
The baseline is known or presumed to have occurred within the prior 7 days.

\begin{equation}
    \text{ Urine volume } \le 0.5 \text{ ml/kg/h for 6 hours}
\end{equation}

Although the KDIGO guidelines provide a standardized definition of AKI, upon detecting AKI, the damage has already occurred due to either high levels of creatinine or accumulation of waste products in the blood.
And the need for early prediction of AKI is demanding to prevent further complications and improve patient outcomes.
That is why my thesis aims to develop a predictive model that can detect the risk of AKI before the serious consequences occur.

While other studies struggled with the subtle nature of AKI, a study by \cite{chen2021nomogram} has developed a nomogram to predict the risk of Acute Kidney Injury for KDA-ICU patients.
This has shed light for the following researches as ICU patients got monitored more frequently than general patients, which could provide more accurate and abundant data for the prediction model.
On the flip side, the study used a deprecated definition of AKI and an older MIMIC version, leading falsely labeled AKI patients and small dataset.
Their best model, achieved an AUC of 0.747, which is a promising result but still has room for improvement.
One of my proposed upgrades is to use the latest MIMIC-IV database and the most recent definition of AKI to improve the model's accuracy.

Another study by \cite{goceri2021predicting} has proposed using predictive models based on a machine learning approach to identify patients with DKA at increased risk of AKI within 1 week of hospitalization in the intensive care unit.
With state-of-the-art machine learning algorithms, the study achieved an AUC of 0.800, which is a significant improvement compared to the previous study.
However, the study did not take into account the changes in patients' indicators over time, which could be crucial for predicting Acute Kidney Injury as the disease progresses. 
Moreover, the study did not provide a detailed process of extracting the features used in the model, making it difficult to replicate the results.
The researchers also use some features that are not available during real treatment process such as patient's lethality rate or continuous renal replacement therapy status of the patients which indicate the patient is already in serious stage of Acute Kidney Injury.
Thus, I planed to remove these features and add frequently occurring measurements as well as propose a new model that can consider the changes in patients' indicators.



% \section{Acute Kidney Injury}
% Acute Kidney Injury (AKI) is a sudden episode of kidney failure or damage that happens within a few hours or days. 
% AKI leads to a buildup of waste products in the blood, making it difficult for the kidneys to maintain the right balance of fluids in the body. 
% This condition can also affect other organs, including the brain, heart, and lungs. Due to its critical nature, AKI requires quick detection and constant management to prevent long-term consequences and improve patient outcomes.

% AKI can result from various causes, typically categorized into three groups: pre-renal, intrinsic renal, and post-renal. 
% Pre-renal causes include conditions that reduce blood flow to the kidneys, such as severe dehydration, blood loss, or low blood pressure. 
% Heart failure or liver cirrhosis can also lead to pre-renal AKI.
% Intrinsic renal causes are conditions that directly damage the kidneys, such as glomerulonephritis, acute tubular necrosis, or interstitial nephritis, often due to nephrotoxic drugs, infections, or autoimmune diseases. % kidney toxic drugs, infections, or autoimmune diseases.
% Post-renal causes involve obstructions that block urine flow out of the kidneys, such as kidney stones, tumors, or an enlarged prostate.
 
% Early detection of AKI is crucial for timely intervention and management, potentially reversing the condition before it causes irreversible kidney damage. 
% The early signs of AKI can be subtle and non-specific, making awareness and prompt diagnosis essential. 
% Clinical evaluation and history are vital, especially for patients with risk factors such as major surgery, critical illness, or existing kidney disease. 
% Symptoms like reduced urine output, swelling in legs, ankles, or around the eyes, fatigue, confusion, nausea, or chest pain should be closely monitored.

% Laboratory tests and biomarkers play a significant role in early detection. 
% Serum creatinine is a key indicator, with even a small rise suggesting kidney dysfunction. 
% Elevated Blood Urea Nitrogen (BUN) levels can also indicate kidney problems, while monitoring urine output can reveal early signs of AKI through oliguria. 

% In conclusion, AKI is a serious medical condition with significant morbidity and mortality. 
% Early detection through clinical vigilance, the use of biomarkers, and imaging studies is essential for prompt management and improving patient outcomes. 
% Preventative strategies and careful monitoring of at-risk individuals can significantly reduce the incidence and impact of AKI. 
% Advances in biomarkers and diagnostic techniques hold promise for even earlier detection and intervention, ultimately improving the prognosis for patients with AKI.


% \section{Diabetic Ketoacidosis}
% Diabetic Ketoacidosis (DKA) is a serious and potentially life-threatening complication of Diabetes Mellitus, characterized by hyperglycemia, ketosis, and metabolic acidosis. 
% DKA commonly occurs in individuals with Type 1 diabetes but can also present in those with Type 2 diabetes, particularly under conditions of severe stress, infection, or inadequate insulin therapy. 

% Diabetic Ketoacidosis arises from a severe deficiency of insulin, leading to hyperglycemia and the accumulation of ketone bodies due to increased fatty acid oxidation. 
% The primary precipitating factors include infection, missed insulin doses, and new-onset diabetes. 
% Without sufficient insulin, glucose cannot enter cells for energy, prompting the body to break down fats as an alternative energy source. 
% This process produces ketones, which are acidic by-products that accumulate in the blood, leading to metabolic acidosis. 
% The hallmark signs of DKA include high blood glucose levels (hyperglycemia), the presence of ketones in the urine and blood (ketonemia), and acidosis.

% It can significantly contribute to the onset and progression of AKI through several mechanisms such as Dehydration and Volume Depletion, Electrolyte Imbalances, Acidosis, Inflammatory Response and Oxidative Stress. 
% Severe dehydration due to osmotic diuresis is a primary factor in DKA. 
% The hyperglycemic state causes an increase in osmotic pressure, leading to significant fluid loss through the kidneys. 
% This profound volume depletion reduces renal perfusion, potentially precipitating pre-renal AKI. 
% If dehydration is not promptly corrected, it can progress to intrinsic renal damage.
% In addition, DKA is associated with significant electrolyte disturbances, including hyperkalemia and hyponatremia. 
% The shifting of potassium from the intracellular to the extracellular space, followed by renal losses, can lead to significant potassium depletion. 
% Electrolyte imbalances can exacerbate renal dysfunction, further contributing to AKI
% The metabolic acidosis seen in DKA can also have direct toxic effects on the kidneys. 
% Acid-base disturbances alter renal blood flow and glomerular filtration rate, leading to further kidney injury. 
% Acidosis also affects the tubular cells, making them more susceptible to damage and necrosis.
% Furthermore, Diabetic Mellitus triggers a systemic inflammatory response and increases oxidative stress. 
% The release of pro-inflammatory cytokines and reactive oxygen species can lead to endothelial dysfunction and tubular injury, thereby promoting AKI.

% In short, DKA is a serious complication of diabetes that can significantly impact renal function and lead to Acute Kidney Injury. 
% The pathophysiological mechanisms linking DKA and AKI include dehydration and volume depletion, electrolyte imbalances, metabolic acidosis, and systemic inflammation. 
% Early recognition and prompt management of DKA are vital to prevent the development of AKI and other complications.

% Correcting hyperglycemia, dehydration, and electrolyte imbalances have gigantic effect on preventing the onset of AKI. 
% Once the risk was detected, the patient can be closely monitored for signs of AKI and treated promptly with Fluid Resuscitation, Insulin Therapy, Electrolyte Management. 
% Fluid replacement help to restore intravascular volume and renal perfusion, while insulin therapy corrects hyperglycemia and suppresses ketogenesis.
% Regular monitoring and correction of electrolyte imbalances are essential to relieve the pressure on the kidneys and prevent further renal damage.


% TODO: Add theoretical knowledge about models




\end{document}