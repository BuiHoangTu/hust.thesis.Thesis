\documentclass[../main.tex]{subfiles}
\begin{document}

\section{Summary}

% Sinh viên nhắc lại các vấn đề mà đồ án đã giải quyết được, cũng như những vấn đề còn tồn đọng của đồ án.  

This thesis explores the application of deep learning models in predicting \gls{AKI} in patients with \gls{DKA}.
AKI, a severe complication that affects about 40\% of \gls{DKA} patients, can significantly increase morbidity and mortality. 
Early prediction of \gls{AKI} is crucial for timely intervention and improving patient outcomes.

The study focuses on developing and comparing multiple deep learning models, including tabular-based and time series-based approaches, to predict the risk of \gls{AKI} within the first 7 days of \gls{ICU} admission for \gls{DKA} patients. 
The models were trained using various features collected 24 hours prior to \gls{ICU} admission, such as demographics, vital signs, comorbidities, testing results, interventions, prognosis, and scoring systems.

Key findings of this research highlight the effectiveness of deep learning models over traditional machine learning methods, particularly in handling the temporal changes in patients' indicators. 
The proposed models demonstrated superior performance, offering a valuable tool for early detection of \gls{AKI} and aiding healthcare professionals in making informed decisions for patient care.

This work contributes to the field of medical technology by leveraging the power of artificial intelligence to address critical healthcare challenges.
With such high accuracy and reliability, the developed models can be integrated into clinical practice to enhance patient care and reduce the burden on healthcare systems.


\section{Suggestion for Future Works }

Although having achieved promising results, this study has several limitations that could be addressed in future research.
Firstly, the dataset used in this study was collected from a single source, which may not be representative of the general population.
Future studies should consider using a more diverse dataset to improve the generalizability of the models.

Secondly, the study focused on predicting \gls{AKI} within the first 7 days of \gls{ICU} admission.
This means \gls{AKI} high risk patients who develop the condition in normal hospital wards are not considered.
Future research could explore the prediction of \gls{AKI} in a broader context, including patients outside the ICU.

Lastly, the study utilized a limited set of features collected 24 hours prior to \gls{ICU} admission.
Which limits the model's ability to capture the full development of the patient's condition.
Future research could aim to promptly predict \gls{AKI} with real-time data, providing more accurate and timely predictions.

\end{document}