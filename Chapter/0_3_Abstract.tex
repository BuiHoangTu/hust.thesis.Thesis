\documentclass[../main.tex]{subfiles}
\begin{document}

\begin{center}
    \Large{\textbf{ABSTRACT}}\\
\end{center}
\vspace{1cm}

\gls{DKA} is a serious acute complications of \gls{DM} that can be life-threatening if not treated promptly.
DKA can lead to other complications such as Cerebral Edema, \gls{AKI}, or in severe cases, kidney failure.
Among them, \gls{AKI} is a common complication, affecting about 40\% of \gls{DKA} patients.
Furthermore, \gls{AKI} increases mortality and morbidity from other diseases, leading to prolonged treatment time in the \gls{ICU}.
This complication has a high recurrence rate during \gls{ICU} treatment and can result in \gls{CKD}.
Usually, upon detected, Acute Kidney Injury has already make a significant impact on the patient's health.

\par Predicting the risk of \gls{AKI} in \gls{DKA} patients is essential for early intervention and treatment, helping minimize negative effects.
Several risk factors of developing \gls{AKI} for \gls{DKA} patients have been investigated over the past decade, including older age, increased glucose, serum uric acid, white blood, \dots
In previous studies, models have been applied to predict the risk of \gls{AKI} in \gls{DKA} patients using logistic regression or machine learning algorithms.
However, according to my knowledge, there is currently no models taking the changes in patients' indicators overtime into account and the power of deep learning is still underexplored.

\par Therefore, in this thesis, I propose a deep learning model to predict the risk of \gls{AKI} in \gls{DKA} patients during their \gls{ICU} stay.
Multiple tabular-based deep learning models and time series-based deep learning models are implemented and compared with traditional machine learning models.
The models are trained to predict if a \gls{DKA} patient will develop \gls{AKI} within the next 7 days after \gls{ICU} admission.

The following features are collected 24 hours prior to the admission time: demographics, vital signs, comorbidities, testing, interventions, prognosis, and scoring systems. 
The models are evaluated based on the \gls{AUC-ROC}. 
The final result show TabPFN model outperforms other models with an AUC-ROC of 0.83 while the time series-based model built on top of \gls{LSTM} is more suitable for hospital settings with Recall of 0.68.

% \begin{flushright}
% \begin{tabular}{@{}c@{}}
% Student\\
% \textit{(Signature and full name)}
% \end{tabular}
% \end{flushright}

\end{document}