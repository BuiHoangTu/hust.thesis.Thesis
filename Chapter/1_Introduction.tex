\documentclass[../main.tex]{subfiles}
\begin{document}


\section{Problem Statement}
\label{sec:dvd}

Diabetic Ketoacidosis is a metabolic complication of Diabetes Mellitus, caused by insulin deficiency and increased counterregulatory hormones.
This results in the body not converting enough sugar into energy and requiring to metabolize triglycerides.
Overtime, this process produces free fatty acids which are converted into ketone bodies in the liver.
The evaluation of FFAs exacerbates insulin resistance and hyperglycemia, creating a malignant cycle.
In Vietnam, the increasing number of people with diabetes has doubled in the past 6 years from 3.53 million people in 2017 to 7 million people. % (viet.gov)
A study conducted in Australia of 8533 DKA patients illustrated that the rate of DKA admission to intensive care unit has increased fivefold over the past decade.
This makes complications of Diabetic Ketosis more common, especially Acute kidney injury.

Acute kidney injury (AKI) is a common complication of hospitalized patients, with a high mortality rate.
It occurs in approximately 30-50\% of ICU patients.
AKI is characterized as a sudden worsen of the kidney's blood filtering function, usually caused by a buildup of waste in the blood that prevents the kidneys from balancing the body's electrolytes.
This complication can develop in just a few hours or a few days.
In most cases, it occurs within 48 hours but can last up to 7 days.
AKI can cause serious complications such as chronic kidney disease or damage to other organs in the body such as the brain, heart and lungs.
Often when acute kidney injury is detected, it has left severe sequelae due to waste accumulation in the blood affecting organs in the body, increasing mortality, treatment time and cost.
In addition, due to damage to the kidneys caused by AKI, patients face a greater susceptibility of disease recurrence, especially in patients with underlying diseases that put pressure on the kidneys such as DKA patients.

Therefore, predicting AKI in advance is an urgent issue in improving patient care, allowing doctors to intervene promptly to stop disease progression.



\section{Background and Problems of Research} 
\label{sec:giaiphap}

Several studies have made significant advances in predicting AKI in DKA-ICU patients using mainly tabular-based machine learning models.
These studies shed light on a new direction for predicting the risk of developing Acute Kidney Injury for DKA-ICU patients alone rather than for all ICU-admitted patients.
In more recent study, the researchers proposed clinical and biological features, including biological indicators such as creatinine, urea, glucose, etc. and clinical indicators such as blood pressure, heart rate, etc., to build an AKI prediction model using machine learning algorithms showed that the model's ability to predict AKI was more optimal than previous models with AUCs of 80\% and 74.7\% respectively.

Although they have shown success in predicting patients with Acute Kidney Injury, these models still have some limitations and shortcomings.
In both models, the data used to train the model is only static data, does not account for indicators changes over time.
This criples the model's performance in the patients whose indicators change drastically after the initial observation of which the data is used to train the model.

In addition, the model uses some unreasonable indicators which were not available before the development of AKI including mortality rate, interventions specific to Acute Kidney Injury.
These features cause data leakage with the model, leading to inaccurate assessments with part of the data.

Furthermore, the dataset only includes data from the MIMIC-IV database, so the patient set is small and not representative of all DKA-ICU patients.

Finally, the method for assessing AKI in patients is not yet complete and updated regularly, so there exist patients falsely labeled in previous studies.

Overall, although there have been noticeable advances in predicting AKI in DKA-ICU patients, there are still many limitations that need to be overcome so that the model become stable, reliable, and suitable to more general usecase.



\section{Research Objectives and Conceptual Framework}

The main goals of this thesis is to find out the most suitable model to predict AKI in DKA-ICU patients based on clinical and biological data from MIMIC-IV database.

To begin, the first objective needs to address the static features used  in current models which require me to delve into the MIMIC-IV database to gain a comprehensive understanding of its data structure including the structure of each table, the relationship between tables and how MIMIC-IV store data.
In this section, it is important to remove indicators that may cause data leakage and add indicators commonly observed in DKA-ICU patients which was ignored in previous studies.

The second goal focuses on designing and implementing a data structures that has ability to accommodate patient metrics' changes over time.
To accomplish this, I choose to save the indicators in a JSON-like structure so a data preprocessing pipeline was built to convert data from tabular form to JSON-like format.

Thirdly, I aim to replicate the AKI tabular-based prediction model in DKA-ICU patients that had been performed in previous studies to ensure that the models in the old study and the new models are evaluated on the same data set.

Finally, I evaluate and compare the predictive ability of different models in predicting Acute Kidney Injury in DKA-ICU patients.
The models compared include XGBoost (Best Model in Previous Studies), new deep learning models with tabular data, and deep learning models with time series data.

\section{Contributions}

This thesis' contributions can be summarize into 4 main points.

Firstly, it proposes the removal of certain data-leaking indices and the addition of commonly measured metrics in DKA-ICU patients to predict AKI. 
By identifying and eliminating indices that contribute to data leakage, the accuracy and reliability of the predictions are enhanced. 
Additionally, the inclusion of commonly measured indices expect to boost models' performance.

Secondly, this thesis updates the method for identifying AKI patients. This improvement involves employing latest algorithms utilized to indicate Acute Kidney Injury, ensuring that the identification process is more accurate.
By enhancing the identification method, the project aims to improve early detection and intervention for patients at risk of AKI, ultimately leading to better outcomes.

Thirdly, it evaluates the predictive capability of tabular-based models with different indicators calculation methods. 
This involves a thorough analysis of various methods for calculating indicators if several ones were measured and their impact on the performance of predictive models. 
By comparing these methods, the project identifies the most effective approaches for accurately predicting AKI, providing valuable insights for future research and clinical practice.

Finally, the main contribution of the thesis lies in developing and evaluating a deep learning model with time-series data.
This innovative approach leverages the power of deep learning to analyze patterns and trends in time-series data, offering a sophisticated tool for predicting AKI.

\section{Organization of Thesis}
The rest of the thesis report is organized into five main chapters.

Chapter 2 provides a comprehensive review of existing literature pertaining to the prediction of Acute Kidney Injury,  which consist of its pathophysiology, risk factors, and clinical outcomes.
By delving into past research, I will explore the various predictive models, risk factors, and early biomarkers identified over the years.
This examination will shed light on how early prediction of AKI has evolved, highlighting key advancements and methodologies that have contributed to improved predictive accuracy.
Understanding these foundational studies will provide valuable insights into the development of current prediction strategies and help identify potential areas for further research and innovation in forecasting AKI.

In Chapter 3, I inspect the Medical Information Mart for Intensive Care IV (MIMIC-IV) database to establish an method to extract the desired data for predicting AKI. 
MIMIC-IV, with its rich repository of de-identified health records, offers a valuable resource for studying Acute Kidney Injury and other critical conditions. 
I will detail the process of navigating the database structure, identifying relevant data fields, and querying the database to retrieve specific datasets. 
Additionally, I will outline the steps involved in building a robust data pipeline to ensure efficient storage, retrieval, and management of the extracted data. 
This pipeline will incorporate best practices for data cleaning, transformation, and integration, facilitating seamless downstream analysis and research. 
By the end of this chapter, a clear framework for leveraging MIMIC-IV data in AKI prediction studies will be established, paving the way for more effective and data-driven medical research.

Chapter 4 details how to build a model to predict Acute Kidney Injury in Diabetic Ketoacidosis patients in the ICU using time series data. 
The foundation of the model leverages TensorFlow, a popular and powerful deep learning library known for its flexibility and scalability. 
Unlike previous models that only utilize data from the first 24 hours of ICU admission, this model is designed to continuously receive and process data up to the present time of the treatment process. 
This continuous data integration allows for real-time predictions and dynamic adjustments, enhancing the model's accuracy and responsiveness. 
In this chapter, I will explain the step-by-step process of designing, training, and validating this time series model, including the selection of appropriate features, handling missing data, and optimizing hyperparameters. 
By employing TensorFlow's advanced capabilities, the model aims to provide more timely and accurate predictions of AKI, thereby improving patient outcomes through earlier interventions and tailored treatment plans.

Finally, in Chapter 5, I will compare the results of the built models and evaluate their predictive ability. 
These models will be assessed using various model evaluation metrics, such as the Area Under the Curve (AUC), precision, accuracy and recall. 
By benchmarking these metrics, I will compare the performance of our models against those from previous studies. 
This comparative analysis will highlight the strengths and weaknesses of each model, providing insights into their effectiveness in predicting acute kidney injury (AKI) in diabetic ketoacidosis (DKA) patients in the ICU. 
Additionally, I will discuss any observed improvements or shortcomings, offering explanations for the differences in performance. 
This chapter aims to provide a comprehensive evaluation, ensuring that the most accurate and reliable model is identified for practical implementation in clinical settings.


\section{Conclusion}
In this chapter, I have presented an overview of the project including the problem, goals and solution orientation along with the project's contributions and the project's layout.
In the next chapters, I will go deeper into each part of the project to solve the problems mentioned in this chapter.


\end{document}